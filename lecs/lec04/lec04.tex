\documentclass[11pt, oneside]{article} 
\usepackage{amsmath, amsthm, amssymb, calrsfs, wasysym, verbatim, bbm, color, graphics, graphicx, geometry}
\usepackage[most]{tcolorbox}
\usepackage{xcolor}
\usepackage{framed}
\usepackage{caption}
\usepackage{subcaption}
%\colorlet{shadecolor}{blue!15}
\graphicspath{ {./figs} }

\geometry{tmargin=.75in, bmargin=.75in, lmargin=.75in, rmargin = .75in}  

\newcommand{\R}{\mathbb{R}}
\newcommand{\C}{\mathbb{C}}
\newcommand{\Z}{\mathbb{Z}}
\newcommand{\N}{\mathbb{N}}
\newcommand{\Q}{\mathbb{Q}}
\newcommand{\Cdot}{\boldsymbol{\cdot}}

%\newtheorem{thm}{Theorem}
%\newtheorem{defn}{Definition}
%\newtheorem{conv}{Convention}
%\newtheorem{rem}{Remark}
%\newtheorem{lem}{Lemma}
%\newtheorem{cor}{Corollary}
%\newtheorem{exa}{Ejemplo}

\newtcbtheorem[auto counter]{eje}%
  {Ejemplo}{fonttitle=\bfseries\upshape, fontupper=\slshape,
     arc=0mm, colback=blue!5!white,colframe=blue!75!black}{Ejemplo}

\newtcbtheorem[auto counter]{alg}%
  {Algoritmo}{fonttitle=\bfseries\upshape, fontupper=\slshape,
     arc=0mm, colback=red!5!white,colframe=red!75!black}{Algoritmo}

\title{Estructuras Hidr\'aulicas [2015961] \\ \textbf{Tema \# 4: Flujo r\'apidamente variado - Estructuras hidr\'aulicas}}
\author{\textbf{Luis Alejandro Morales, Ph.D}\\ \vspace{0.4cm} Profesor Asistente \\ Universidad Nacional de Colombia-Bogot\'a\\Facultad de Ingenier\'ia \\ Departamento de Ingenieria Civil y Agr\'icola}
%\date{Periodo 2022-II}
\date{}

\begin{document}

\maketitle
\tableofcontents

%\vspace{.25in}

%%%%%%%%
\section{Introducci\'on} % From Chau 
El estudio de estructuas hidr\'aulicas hace referencia al an\'alisis de flujo r\'apidamente variado (FRV) en canales. Este tipo de fluidos presente en diferentes estructuras hidr\'aulicas es caracterizado por altos gradientes de presi\'on que hace que el comportamiento hidroest\'atico de presiones no se cumpla. Un ejemplo de ello es el \emph{resalto hidr\'aulico} en donde se tiene un cambio de r\'egimen supercr\'itico a subcr\'itico por cambios en la geometr\'ia del canal. Teniendo en cuenta que la distribuci\'on hidroest\'atica de presiones no es aplicable para FRV ya que las l\'ineas de flujo no son aproximadamente paralelas, el an\'alisis de estos flujos se ha realizado desde aproximaciones emp\'iricas analizando diferentes fen\'omenos y estructuras hidr\'aulicas por separado. Para estos an\'alisis emp\'iricos se han empleado las aproximaciones de \emph{Boussinesq} y \emph{Fawer} en donde la velocidad vertical sigue una ley lineal y exponencial, respectivamente, en donde la velocidad en el fondo es 0. 

El FRV ocurre en distancias cortas por lo que la p\'erdida de energ\'ia es despreciable. Por ser flujos r\'apidos, esto genera la creaci\'on de vortices y ondulaciones de la superficie del agua que aparecen y desaparecen rapidamente. Esto hace que sea dif\'icil determinar valores medios para la velocidad y la profundidad en una secci\'on.  


\section{Leyes de conservaci\'on para flujo r\'apidamente variado}
Para el an\'alisis de las leyes de conservaci\'on de la masa, de la energi\'ia y de la cantidad de movimiento, consideremos las condiciones de flujo en un canal horizontal el cual posee un escal\'on en tres secciones transversales (ver figura~\ref{fig71}). Note que la secci\'on 1 y 3 se considera que el flujo alcanza condiciones uniformes mientras que en la secci\'on 2, justo despu\'es del escal\'on, hay una separaci\'on del flujo y por lo tanto la distribuci\'on de velocidades es como se muestra en la figura. Esto implica que en las secciones 1 y 3, las presiones son hidroest\'aticas, lo contrario a la distribuci\'on de presiones en la secci\'on en 2. 

% Chau fig 7-1
\begin{figure}[h]
    \centering
    \includegraphics[width=0.8\linewidth]{fig71.png}
    \caption{Cambio abrupto en el fondo del canal.}
    \label{fig72}
\end{figure}

\subsection{Conservaci\'on de la masa}
Por definici\'on, el flujo volum\'etrico o caudal a trav\'es de una secci\'on se define como:
\begin{equation}
    Q = \int_A \mathbf{u} \cdot d\mathbf{A}
\label{eq1}
\end{equation}

donde $\mathbf{u}$ es el campo vectorial de velocidades, $d\mathbf{A}$ es el vector normal de area infinitesimal y $A$ es el area de la secci\'on. Si el flujo es aproximadamente paralelo, lo cual ocurre cuando existe presiones hidroest\'aticas, el angulo formado entre $\mathbf{u}$ y $d\mathbf{A}$ es cero, y si la velocidad es casi uniforme (velocidad media $V$), como es el caso de las secciones 1 y 2 en la figura~\ref{fig71}, la ecuaci\'on~\ref{eq1} se convierte en:
\begin{equation}
    Q = V \int_A  dA = QA
\label{eq2}
\end{equation}

La ecuaci\'on~\ref{eq1} es conocida como la ecuaci\'on de continuidad por lo que, para flujo permanente, $Q_1 = Q_3$ o $A_1 V_1  = A_3 V_3$. 

Analizando el la distribuci\'on de velocidades en 2, es notorio que existe un flujo de reversa por lo que expresar la velocidad en terminos de la velocidad media $V$ no es posible. Esto implica que el c\'alculo de $Q$ en la secci\'on 2 a partir de la ecuaci\'on~\ref{eq2} solo es posible si la distribuci\'on de velocidad representada por el vector $\mathbf{u}$  es conocida. 


\subsection{Conservaci\'on de la cantidad de movimiento}
Por definici\'on, el flujo de cantidad de movimiento  a trav\'es de una secci\'on $A$ en direcci\'on $x$ se define como: 
\begin{equation}
    m_x = \rho \int_A v_x \left( v dA \right)
\label{eq3}
\end{equation}

donde $v_x$ es la componente de la velocidad en $x$. Para evaluar la ecuaci\'on~\ref{eq3} es necesario conocer las componented en $y$ y $z$ de la velocidad. Ecuciones similares se pueden obtener para $m_y$ y $m_z$.

Considerando una velocidad uniforme en la secci\'on, el flujo de cantidad de movimiento se puede calcular como:
\begin{equation}
    m = \rho Q \left( \beta V \right)
\label{eq4}
\end{equation}
en donde $\beta$ es un coeficiente de cantidad de movimiento el cual corrige $m$ teniendo en cuenta la distribuci\'on no uniforme de la velocidad en la secci\'on. Analisando la figura~\ref{71}, la ecuaci\'on~\ref{eq4} puede ser utilizada para estimar $m$ en las secciones 1 y 3 pero no en la secci\'on 1. 

La fuerza actuante sobre una secci\'on puede ser obtenida a partir de la ecuaci\'on~\ref{eq4} siempre y cuando la presi\'on en la secci\'on sea hidroest\'atica como es el caso de las secciones 1 y 3. Para el caso de la secci\'on 2 en donde no es posible determinar la fuerza  a menos que la funci\'on de distribuci\'on de presiones sea conocida, lo cual puede lograrse a trav\'es de mediciones en el labotorio.  Esto hace que la aplicaci\'on de la cantidad de movimiento en FRV sea complicada. 


\subsection{Conservaci\'on de la energ\'ia}
La energ\'ia total en una secci\'on de flujo en donde la presi\'on es hidroest\'atica se expresa como:
\begin{equation}
    H = z + y + \alpha \frac{V^2}{2g}
\label{eq5}
\end{equation}

En secciones en donde la presi\'on no es hidroest\'atica la ecuaci\'on anterior no aplica. En terminos generales, el flujo de energ\'ia en una secci\'on se expresa como:
\begin{equation}
    P = \rho g \int_A \left( z + \frac{p}{\gamma} + \frac{v^2}{2g} \right) v dA
\label{eq6}
\end{equation}

Para solucionar la ecuaci\'on anterior, la distribuci\'on de velocidades ($v$) y de presiones ($p$) debe ser conocida.

En resumen, no es posible utilizar el concepto de velocidad media o de presiones hidroest\'aticas para FRV, por lo que es necesario conocer la distribuci\'on de velocidades ($v$) y de presiones ($p$). Esto quiere decir que las leyes de conservaci\'on, como han sido comunmente aplicadas, no son posibles para FRV. Distribuciones teoricas de la velocidad como la de Bousinessq y la de Fawer pueden ser empleadas. Sin embargo en un canal en donde se presentan FGV y FRV, es preferible hacer el analisis en regiones en donde FRV no este presente. 

\section{Transiciones en canales}

Una transici\'on es un cambio local en las caracter\'isticas del canal, usualmente, cambio en el \'area , forma o direcci\'on del canal, que resulta en un cambio de estado en el flujo. Transiciones t\'ipicas son las expansiones, las contracciones y las curvas. Aquellas transiciones en donde se obtenga una relaci\'on $y=f(Q)$ se conoce como un \emph{control}; all\'i ocurre la profundidad cr\'itica. Un \emph{control artificial} ocurre en  la entrada de \emph{aliviaderos} o en la cresta de \emph{vertederos}. Por otro lado un \emph{control natural} es aquel que se presenta en la descarga libre de un canal. 

A parte de servir como estructuras para cambiar el alineamiento y/o la secci\'on transversal de un canal, las transiciones son tambi\'en diseñadas para minimizar la p\'erdida de energ\'ia, para disipar energ\'ia o para reducir la velocidad del flujo y evitar erosi\'on. Teniendo en cuenta la relacion un\'ivoca $y=f(Q)$, las transiciones tambi\'en se usan para medir el $Q$. 

\subsection{Caracter\'isticas generales}
El diseño y construcci\'on de transiciones require:
\begin{itemize}
\item Con el fin de minimizar costos y por facilidad constructiva, las transiciones deben ser simples y pueden tener fronteras discontinuas.
\item Si es necesario minimizar las p\'erdidas de energ\'ia, la transici\'on debe ser gradual y no deben existir discontinuidades en los bordes. Diseños de este tipo previenen la formaci\'on de remolinos y la separaci\'on del flujo reduciendo la posibilidad de \emph{cavitaci\'on}.
\item Las transiciones causan aceleraci\'on o desaceleraci\'on del flujo en una distancia relativamente corta y dominan el movimiento por encima de los esfuerzos cortantes en las fronteras. Esto implica que el flujo no se 1D, que las lineas de flujo se curven y que exista separaci\'on del flujo. 
\item En flujo con fuerte aceleraci\'on vertical, la velocidad y la presi\'on no solo cambian en la direcci\'on del flujo si no  tambi\'en en la vertical lo cual da lugar a flujos 2D o 3D.
\item La p\'erdida de energ\'ia en una transici\'on es despreciable y el flujo se asume irrotacional. 
\item Durante el diseño de una transici\'on, es necesario evitar la ocurrencia de ca\'idas de presi\'on (e.g. cavitaci\'on) o aumentos de presi\'on que pueden causar vibraciones e inestabilidades. 
\item Para el an\'alisis de flujo en transiciones, la distribuci\'on de velocidades es generalmente no uniforme y es posible tener velocidades negat\'ivas en partes de la secci\'on. Esto hace que sea dif\'icil calcular, por ejemplo, la energ\'ia total en la secci\'on.
\end{itemize}

\subsection{Flujo subcr\'itico}
\subsubsection{Expansiones}
Una expansi\'on ocurre por un incremento en el ancho del canal, una ca\'ida del fondo o una combinaci\'on de ambos (ver figura~\ref{fig72})
% Chau fig 7-2
\begin{figure}[h]
    \centering
    \includegraphics[width=0.8\linewidth]{fig72.jpeg}
    \caption{Expansi\'on en un canal.}
    \label{fig72}
\end{figure}

Las expansiones pueden ser abruptas (ver figura~\ref{fig72}) o graduales. Estas estructuras est\'an presentes en canales, descargas, sifones y acueductos. El diseño de expansiones requiere la selecci\'on de la forma para evitar separaci\'on de flujo y minimizar p\'erdida de energi\'a. Los criterios de diseño de expansiones se apoya en an\'alisis experimentales. Algunos hallasgos:
\begin{itemize}
    \item Se ha encontrado que el flujo aguas abajo de la expansi\'on es asim\'etrico cuando $\frac{B_2}{B_1} \leq 1.5$, donde $B_1$ es el ancho aguas arriba y $B_2$ es el ancho aguas abajo de la expansi\'on. 
    \item La forma de una linea de flujo a trav\'es de una expansi\'on se puede representar por la siguiente ecuaci\'on:
    \begin{equation}
        \frac{B_x - B_1}{B_2 - B_1} = \frac{x}{L} \left[ 1- \left(1-\frac{x}{L}\right)^m \right]
        \label{eq1}
    \end{equation}
    en donde $B_x$ es dos veces la distancia desde el eje de simetr\'ia de la expansi\'on a la l\'inea de flujo mas externa, $x$ es la distancia horizontal a lo largo del eje de simetr\'ia desde donde inicia la expansi\'on, $L$ es la distancia horizontal a lo largo del eje de simetr\'ia donde la linea de flujo toca la frontera y $m$ es un exponente que varia como $0.6\leq m \leq 0.66$. 
    
    Al diseñar la forma de una expansi\'on siguiendo la ecuacion anterior, se garantiza la menor separaci\'on de flujo y se minimizan las p\'erdidas de energ\'ia. 
\end{itemize}

Consideremos la expansi\'on repentina mostrada en la figura~\ref{fig73} en donde se cambia de manera abrupta de un ancho $B_1$ a $B_2$.
% Chau fig 7-3
\begin{figure}[h]
    \centering
    \includegraphics[width=0.8\linewidth]{fig73.jpeg}
    \caption{Expansi\'on en un canal.}
    \label{fig73}
\end{figure}

Si se asume que: 1) $E_2 = E_1$, 2) $F_{s_1} = F_{s_2}$, 3) $y_1 = y_2$, 4) $B_2 \approx B_1$ y 5) $Fr_1^n \approx 0$ para $n \ge 4$, se puede demostrar que:
\begin{equation}
    E_1 - E_3 = \frac{V_1^2}{2g}\left[ \left( 1- \frac{B_1}{B_2}\right)^2 + 2Fr_1^2 \left(B_2 - B_1\right)\frac{B_1^3}{B_2^4}\right]
    \label{eq2}
\end{equation}
En la ecuaci\'on anterior, el segundo termino dentro de los parentesis es despreciables si $Fr_1 < 0.5$ o si $B_2 / B_1 > 1.5$. En muchas aplicaciones, la condici\'on  $B_2 / B_1 > 1.5$ se cumple. Para el caso en que $B_2 / B_1 < 1.5$ las p\'erdidas de energ\'ia en la expansion son despreciables. 

En terminos generales las p\'erdidas de energ\'ia en una expansi\'on subita, se calcula como:
\begin{equation}
    H_l = \frac{\left( V_1 - V_3 \right)^2}{2g}
    \label{eq3}
\end{equation}
An\'alisis experimentales han demonstrado que la ecuaci\'on anterior sobre-estiman las p\'erdidas de energ\'ia en cerca de un 10\%.

En expansiones graduales cuyos bordes cambian como 4H:1V, las p\'erdidas de energ\'ia son:
\begin{equation}
    H_l = 0.3\frac{\left( V_1 - V_3 \right)^2}{2g}
    \label{eq4}
\end{equation}
Para transiciones m\'as graduales, las p\'erdidas no son significativamente menores a las estimadas en la ecuaci\'on~\ref{eq4}, sin embargo, los costos pueden aumentar sustancialmente.

\subsubsection{Contracciones}
Las contracciones en un canal se presentan cuando el ancho del canal se reduce, cuando el fondo del canal aumenta o como una combinaci\'on de ambas (ver figura~\ref{fig74}). 
% Chau fig 7-4
\begin{figure}[h]
    \centering
    \includegraphics[width=0.8\linewidth]{fig74.jpeg}
    \caption{Contracci\'on en un canal.}
    \label{fig74}
\end{figure}
Existen contracciones abruptas y graduales. An\'alisis experimentales han demonstrado que las p\'erdidas de energ\'ia en una contracci\'on son menores que las presentes en una expansi\'on, y son iguales a:
\begin{equation}
    H_l = 0.23\frac{V_3^2}{2g}
    \label{eq5}
\end{equation}
La ecuaci\'on~\ref{eq5} aplica para contracciones abruptas. Para contracciones con bordes redondeados, la p\'erdida de energ\'ia se calcula como:
\begin{equation}
    H_l = 0.11\frac{V_3^2}{2g}
    \label{eq5}
\end{equation}
Note que $V_3$ es la velocidad en la seccion de flujo de mayor contracci\'on (aguas abajo de la contracci\'on) en donde la velocidad es c\'asi uniforme. Otros autores han econtrado que la perdidas de energ\'ia se calculan como:
\begin{equation}
    H_l = C\frac{V_3^2}{2g}
    \label{eq5}
\end{equation}
donde $C=0.35$ para bordes cuadrados y $C=0.18$ para bordes redondeados. 

Contracciones muy fuertes pueden causar condiciones criticas del flujo en la contracci\'on o que la energ\'ia aguas arriba no sea suficiente para pasar a trav\'es de la contracci\'on. 

\subsection{Supercritical flow}
El flujo supercr\'itico a trav\'es de transiciones puede ser problem\'atico porque es com\'un la formaci\'on de ondas de choque en la superficie del flujo. Las ondas de choque son cambios repentinos de la profundidad y de la velocidad que se producen por cambios dr\'asticos en las condiciones de flujo. 

Para analisar las ondas de choque imaginemos un observador que viaja sobre un fluido cuya velocidad es $V$  y causa una perturbaci\'on (e.g. cambios en el alineamiento del canal, irregularidades en en la superficie de la pared, etc). La celeridad de una onda, se define como $c$, la cual es la velocidad relativa con la cual viaja la perturbaci\'on en el flujo. Dependiendo de la magnitud de $V$ y $c$ existen tres situaciones posibles  que se observan en la figura~\ref{fig75}.
% Chau fig 7-5
\begin{figure}[h]
    \centering
    \includegraphics[width=0.8\linewidth]{fig75.jpeg}
    \caption{Propagaci\'on de una perturbaci\'on.}
    \label{fig75}
\end{figure}
Analizando el caso $V>c$, es posible encontrar una relaci\'on entre $V$, $c$ y $\beta$:
\begin{equation}
    \sin \beta = \frac{A_1 D_1}{A_1 A_4} = \frac{c \Delta t}{V \Delta t} = \frac{c}{V}
    \label{eq6}
\end{equation}

En el caso de ondas largas de pequeña amplitud $c = \sqrt{gy}$, donde $y$ es la profundidad de flujo. Reemplazando en la ecuaci\'on~\ref{eq6}, se tiene:
\begin{equation}
    \sin \beta =  \frac{c}{V} = \frac{1}{Fr}
    \label{eq6}
\end{equation}

\subsubsection{Resalto hidr\'aulico oblicuo}
En el caso de un flujo supercr\'itico que pasa a trav\'es de una contracci\'on gradual en un canal rectangular, se generan ondas de gran magnitud debido a la deflexi\'on de las lineas de flujo hacia adentro dando lugar a la ocurrencia de resaltos hidr\'aulicos oblicuos (ver figura~\ref{fig76}). 
% Chau fig 7-6
\begin{figure}[h]
    \centering
    \includegraphics[width=0.8\linewidth]{fig76.jpeg}
    \caption{Resalto hidr\'aulico oblicuo.}
    \label{fig75}
\end{figure}

Analizando la figura~\ref{fig76}, las velocidades tangenciales antes y despues de la contracci\'on (frente de onda) deben ser iguales:
\begin{equation}
   V_1 \cos \beta = V_2 \cos \left(\beta - \Delta \theta \right)
    \label{eq6a}
\end{equation}

donde $V_1$ es la velocidad antes de la contracci\'on, $V_2$ es la velocidad en la contracci\'on, $\Delta \theta$ es el angulo de la contracci\'on y $\beta$ es el anue forma el )rente de onda con la horizontal. 

A partir de la ecuacion de continuidad y de las velocidades perpendiculares al frente de onda, se tiene:
\begin{equation}
   y_1 V_1 \sin \theta = y_2 V_2 \sin \left(\beta - \Delta \theta \right)
    \label{eq7}
\end{equation}

De la conservaci\'on de cantidad de movimiento para la velocidad perpendicular $V_1 \sin \beta$, se tiene:
\begin{equation}
    \frac{V_1 \sin^2 \beta}{gy_1} = \frac{1}{2}\frac{y_2}{y_1}\left( \frac{y_2}{y_1} + 1 \right)
    \label{eq8}
\end{equation}
De esta ecuaci\'on se tiene:
\begin{equation}
    \sin \beta = \frac{1}{Fr_1}\sqrt{\frac{1}{2}\frac{y_2}{y_1}\left( \frac{y_2}{y_1} + 1 \right)}
    \label{eq9}
\end{equation}
Note que para ondas de pequeña amplitud, la ecuaci\'on~\ref{eq9} se convierte en la ecuaci\'on~\ref{eq6}.

Dividiendo la ecuaci\'on~\ref{eq7} por la ecuaci\'on~\ref{eq6a}, se tiene:

\begin{equation}
    \frac{y_2}{y_1} = \frac{\tan \beta}{\tan \left(\beta-\Delta \theta \right)}
    \label{eq10}
\end{equation}
Sustituyendo $y_2 = y_1 + \Delta y$, donde $\Delta y$ es la altura de la onda, en la ecuaci\'on~\ref{eq10}, se tiene:
\begin{equation}
    \frac{\Delta y}{y} = \frac{\sec^2 \beta \tan \Delta \theta}{\tan \beta - \tan \Delta \theta}
    \label{eq11}
\end{equation}

Para valores pequeños de $\Delta \theta$, $\tan \theta \approx \Delta \theta$ y $\tan \Delta \theta$ es muy pequeño con respecto a $\tan \beta$. Aplicando l\'imites cuando $\Delta \theta \rightarrow 0$, la ecuaci\'on~\ref{eq12} se convierte en:
\begin{equation}
    \frac{dy}{d\theta} = \frac{2y}{\sin 2\beta}
    \label{eq12}
\end{equation}
Combinando las ecuaciones ~\ref{eq6} y ~\ref{eq12}, se tiene:
\begin{equation}
    \frac{dy}{d\theta} = \frac{V^2}{g}\tan \beta
    \label{eq13}
\end{equation}
La ecuaci\'on~\ref{eq13} define la variaci\'on de la profundidad de flujo en la transici\'on. Esta ecuaci\'on indica que $y$ cambia en funci\'on de $\theta$ a lo largo del frente de onda oblicuo.

Teniendo en cuenta que las p\'erdidas de energ\'ia pueden son despreciables, la velocidad se puede expresar como $V = \sqrt{2g\left(E-y\right)}$. Reemplazando esta expresi\'on en la ecuaci\'on~\ref{eq13} y utilizando la ecuaci\'on~\ref{eq6}, se tiene:
\begin{equation}
    \frac{dy}{d\theta} = \frac{2\left(E-y\right)\sqrt{y}}{\sqrt{2E-3y}}
    \label{eq14}
\end{equation}

Integrando la ecuaci\'on~\ref{eq14} y sustituyendo $E$ en terminos de $y$ y de $Fr$, se tiene:
\begin{equation}
   \theta = \sqrt{3}\tan^{-1} \frac{\sqrt{3}}{\sqrt{Fr^2 - 1}} - \tan^{-1} \frac{1}{\sqrt{Fr^2 -1}} + \theta_0
    \label{eq15}
\end{equation}
donde $\theta_0$ es la constante de integraci\'on la cual se obtiene sustituyendo $\theta = 0$ para la profundidad $y = y_1$. Esta ecuaci\'on estima los cambios en las profundidades en la transici\'on debido a cambios de $\theta$. 



% REFERENCES
\bibliographystyle{plain} % We choose the "plain" reference style
\bibliography{refs} % Entries are in the refs.bib file

\end{document}
