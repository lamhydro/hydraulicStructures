\documentclass[a4paper, 11pt]{article}
\usepackage{comment} % enables the use of multi-line comments (\ifx \fi) 
\usepackage{lipsum} %This package just generates Lorem Ipsum filler text. 
\usepackage{fullpage} % changes the margin
\usepackage[a4paper, total={7in, 10in}]{geometry}
\usepackage{amsmath}
\usepackage{amssymb,amsthm}  % assumes amsmath package installed
\newtheorem{theorem}{Theorem}
\newtheorem{corollary}{Corollary}
\usepackage{graphicx}
\usepackage{tikz}
\usetikzlibrary{arrows}
\usepackage{verbatim}
\usepackage[numbered]{mcode}
\usepackage{tikz}
\usetikzlibrary{arrows}
\usepackage{caption}
\usepackage{float}
\usepackage{tikz}
\usepackage{cancel}
\usepackage[table,xcdraw]{xcolor}
\usetikzlibrary{shapes,arrows}
\usetikzlibrary{arrows,calc,positioning}

    \tikzset{
        block/.style = {draw, rectangle,
            minimum height=1cm,
            minimum width=1.5cm},
        input/.style = {coordinate,node distance=1cm},
        output/.style = {coordinate,node distance=4cm},
        arrow/.style={draw, -latex,node distance=2cm},
        pinstyle/.style = {pin edge={latex-, black,node distance=2cm}},
        sum/.style = {draw, circle, node distance=1cm},
    }

    \tikzset{every picture/.style={line width=0.75pt}}
\usepackage{mdframed}
\usepackage[shortlabels]{enumitem}
\usepackage{indentfirst}
\usepackage{hyperref}
\usepackage{wrapfig}
\usepackage{standalone}
\setlength{\parindent}{0pt}


\usepackage{listings}

\definecolor{codeblue}{RGB}{33,135,180}
\definecolor{codegray}{rgb}{0.5,0.5,0.5}
\definecolor{codepurple}{rgb}{0.58,0,0.82}
\definecolor{backcolour}{rgb}{0.95,0.95,0.92}

\lstdefinestyle{mystyle}{
    backgroundcolor=\color{backcolour},   
    commentstyle=\color{codeblue},
    keywordstyle=\color{magenta},
    numberstyle=\tiny\color{codegray},
    stringstyle=\color{codepurple},
    basicstyle=\ttfamily\footnotesize,
    breakatwhitespace=false,         
    breaklines=true,                 
    captionpos=b,                    
    keepspaces=true,                 
    numbers=left,                    
    numbersep=5pt,                  
    showspaces=false,                
    showstringspaces=false,
    showtabs=false,                  
    tabsize=2
}

\lstset{style=mystyle}
    
\renewcommand{\thesubsection}{\thesection.\alph{subsection}}

\newenvironment{ejemplo}[2][Ejemplo]
    { \begin{mdframed}[backgroundcolor=gray!20] \textbf{#1 #2} \\}
    {  \end{mdframed}}

% Define solution environment
\newenvironment{solution}
    {\textit{Solution:}}
    {}

\renewcommand{\qed}{\quad\qedsymbol}
%%%%%%%%%%%%%%%%%%%%%%%%%%%%%%%%%%%%%%%%%%%%%%%%%%%%%%%%%%%%%%%%%%%%%%%%%%%%%%%%%%%%%%%%%%%%%%%%%%%%%%%%%%%%%%%%%%%%%%%%%%%%%%%%%%%%%%%%
\begin{document}

\renewcommand{\theenumi}{\alph{enumi})}
%Header-Make sure you update this information!!!!
\noindent
%%%%%%%%%%%%%%%%%%%%%%%%%%%%%%%%%%%%%%%%%%%%%%%%%%%%%%%%%%%%%%%%%%%%%%%%%%%%%%%%%%%%%%%%%%%%%%%%%%%%%%%%%%%%%%%%%%%%%%%%%%%%%%%%%%%%%%%%

\begin{minipage}{3.5cm}
    \includegraphics[width=3.5cm]{Images/logounal.png} % LOGOTIPO
\end{minipage} 
\hspace{10pt}
\begin{minipage}{0.75\textwidth}

    {\Large \textbf{Flujo Crítico} \hspace{6cm} \textbf{}}\vspace{1ex}
    
    \textit{Apuntes de clase}\hspace{8.25cm} Grupo 1\\
    \normalsize Departamento de Ingeniería Civil y Agrícola \hspace{3.6cm} 2024-2S\\
    Docente: Luis Alejandro Morales Marín\hspace{1.9cm}Estructuras Hidráulicas\\
    \noindent\rule{5in}{2.8pt}
\end{minipage} \vspace{10pt} 



\vspace{1ex}

El flujo crítico se presenta cuando la energía en el flujo es igual a la energía crítica $(E_c)$, es decir:

\begin{equation}
    E_c=Ec \quad \rightarrow \quad y=y_c
    \nonumber
\end{equation}

y la profundidad a la que se transporta un caudal determinado $Q$ es $y=y_c$.

\section{Canal Rectangular}

\subsection{Energía Específica}

La energía específica en un canal de baja pendiente es:

\begin{equation}
    E=y+\alpha \dfrac{Q^{2}}{2gA^{2}}
    \label{ecu:ener1}
\end{equation}

para un canal rectangular de ancho $b$ y teniendo que $\alpha=1$, reemplazando en la ecuación \ref{ecu:ener1} tenemos:

\begin{equation}
    E=y+\dfrac{Q^{2}}{2gb^{2}y^{2}}
\end{equation}

En términos del caudal unitario definido cómo $q=\dfrac{Q}{b}$, se tiene que:
\begin{equation}
    E=y+\dfrac{q^{2}}{2gy^{2}}
\end{equation}

Partiendo de los conceptos de cálculo diferencial se conoce que los puntos mínimos o máximos de una función se presentan cuando $\dfrac{dF}{dx}=0$ para una función $f(x)$, por esto si deseamos conocer el punto de inflexión se tiene que $\dfrac{dE}{dy}=0$:

\begin{equation}
    \begin{aligned}
       \dfrac{dE}{dy}=1-\dfrac{2q^{2}}{2gy^{3}}&=0 \qquad y^{3}=\dfrac{q^{2}}{g} \\
       y=&\left(\dfrac{q^{2}}{g}\right)^{\frac{1}{3}}=y_c
    \end{aligned}
    \label{ecu:ycubo}
\end{equation}

Esta profundad $y$ para la cual $E$ es mínima se denomina la \textbf{profundidad crítica} $y_c$. \vspace{1ex}

A partir de $\dfrac{dE}{dy}$, no es posible determinar si la función tiene un mínimo o un máximo en $y_c$, derivando nuevamente se tiene:

\begin{equation}
    \dfrac{d^{2}E}{dy^{2}}=0+\dfrac{3q^{2}}{gy^{4}}=\dfrac{3q^{2}}{gy^{4}}
\end{equation}

Como $\dfrac{d^{2}E}{dy^{2}}$ es siempre positiva para cualquier valor de $y$ dado $q$, $E$ es mínima para $y_c$. \vspace{1ex}

De la ecuación \ref{ecu:ycubo} se tiene que:

\begin{equation}
    y_{c}^{3}=\dfrac{q^{2}}{g}\rightarrow y_{c}=\dfrac{Q^{2}}{gb^{2}y_{c}^{2}} \rightarrow \dfrac{y_{c}}{2}=\dfrac{v_c^{2}}{2g}
\end{equation}

Entonces la profundidad crítica es 2 veces la cabeza de energía cinemática.

Reemplazando en la ecuación de energía específica:

\begin{equation}
    E=y_{c}+\dfrac{y_{c}}{2}=\dfrac{3}{2}y_{c}\rightarrow y_{c}=\dfrac{2}{3}E
\end{equation}

Luego la profundidad crítica es $\frac{2}{3}$ de la energía específica, como:

\begin{equation}
    \dfrac{v_c^{2}}{2g}=\dfrac{y_{c}}{2}\rightarrow\dfrac{v_c^{2}}{gy_{c}}=1 \rightarrow Fr^{2}=1 \rightarrow Fr=1
\end{equation}

Note que el número de Froude es igual a 1 cuando el flujo es \textbf{crítico.}

\subsection{Caudal unitario}

Para determinar los cambios de $q$ con respecto a $y$ para un valor dado de $E$, se tiene:
\begin{equation}
    \begin{aligned}
      E=q+\dfrac{q^{2}}{2gy^{2}}&\rightarrow2gy^{2}(E-y)=q^{2} \\
      q^{2}&=2gq^{2}E-2gq^{3}
    \end{aligned}
    \label{ecu:q2menos}
\end{equation}
 
De la ecuación \ref{ecu:q2menos} tenemos que para $y=0\rightarrow q=0$ al igual que cuando $y=E$. Derivando la ecuación para determinar los valores de $y$ para que $q$ sea $q_{max}$ o $q_{min}$, se tiene:

\begin{equation}
    \begin{aligned}
        \dfrac{d(q^{2})}{dy}&=\dfrac{d(2gEy^{2}-2gy^{3})}{dy} \\
        2q \dfrac{dq}{dy}&=2gE2y-2g3y^{2}\\
        q \dfrac{dq}{dy}&=2gEy-3gy^{2} \rightarrow q \dfrac{dq}{dy}=gy(2E-3y)
    \end{aligned}
    \label{ecu:largaq}
\end{equation}

Igualando la ecuación \ref{ecu:largaq} a 0, se tiene que: $gy(2E-3y)=0$

\begin{equation} 
    \left.\begin{aligned} y&=\dfrac{2}{3}E \\
    y&=0\end{aligned} \right \} \quad \text{Raices} 
\end{equation}

Entonces se puede deducir que $y=\dfrac{2}{3}E\rightarrow y=y_{c}$ \vspace{2ex}

Para determinar si se trata de un máximo o de un mínimo se tiene:

\begin{equation}
    \begin{aligned}
        \left(\dfrac{dq}{dy}\right)&^{2}+q\dfrac{d^{2}q}{dy^{2}}=2gE-6gy \\
        \text{Si} \quad \dfrac{dq}{dy}&=0 \quad \text{para} \quad y=\dfrac{2}{3}E \\
        q\dfrac{d^{2}q}{dy^{2}}=&2gE-4Eg=-2Eg\\
        \dfrac{d^{2}q}{dy^{2}}=\dfrac{-2Eg}{q}&\rightarrow\dfrac{d^{2}q}{dy^{2}} \quad \text{es siempre negativo}
    \end{aligned}
\end{equation}

Esto quiere decir que $q$ es máximo cuando $y=y_{c}$. Sustituyendo en la ecuación para $q=f(y)$ 

\begin{equation}
    \begin{aligned}
        q^{2}_{max}=2g\dfrac{4}{9}E^{3}-2g\dfrac{8}{27}E^{3}\\
        q^{2}_{max}=2g\left(\dfrac{12E^{3}-8E^{3}}{27}\right)=\dfrac{8gE^{3}}{27}
    \end{aligned}  
\end{equation}

La función $q=(2gEy^{2}-2gy^{3})=f(y)$ es una familia de curvas de la forma. \vspace{3ex}


\textcolor{red}{\textbf{(PRIMERA GRÁFICA)}} \vspace{1ex}

% \begin{figure}[H]
%     \centering
%     \includegraphics[width=0.6\linewidth]{}
%     \caption{Familia de curvas}
%     \label{fig:familiacurvas}
% \end{figure} 

\subsection{Fuerza específica}

La fuerza específica es $Fr=\bar{z}A+\beta\dfrac{Q^{2}}{gA}$ sabiendo que para un canal rectangular el ancho es $b$ y $\beta=1$, se tiene:

\begin{equation}
    F_{s}=\dfrac{y^{2}b}{2}+\dfrac{Q^{2}}{gby} \rightarrow \dfrac{y^{2}b}{2}+\dfrac{bq^{2}}{gy}
\end{equation}

La fuerza específica por unidad de ancho

\begin{equation}
    \dfrac{F_{s}}{b}=\dfrac{y^{2}}{2}+\dfrac{q^{2}}{gy}
\end{equation}

Tenemos que $F_{s}=f(y)$ para determinar los puntes de inflexión, derivamos $\dfrac{dF_{s}}{dy}$, entonces:

\begin{equation}
    \begin{aligned}
        \dfrac{dF_{s}}{dy}=&y-\dfrac{q^{2}}{gy^{2}}\\
        \text{haciendo} \quad \dfrac{dF_{s}}{dy}=0 \qquad &y=\dfrac{y^{2}}{gy^{2}}\rightarrow y=\dfrac{v^{2}}{g}
    \end{aligned}
    \label{ecu:V}
\end{equation}

La ecuación \ref{ecu:V} es válida cuando el flujo es crítico. Para determinar si el punto es un máximo o un mínimo, se tiene qué: $\dfrac{d^{2}F_{s}}{dy^{2}}=1+\dfrac{2q^{2}}{gy^{3}}$ \vspace{1ex}

El valor de $\dfrac{dF_{s}}{dy}=1+\dfrac{2q^{2}}{gy^{3}}$ es siempre positivo por lo que la fuerza específica es mínima a la profundidad crítica.

\section{Canal no rectangulares (Prismáticos)}

\subsection{Energía específica}

Partiendo de la ecuación de energía específica:

\begin{equation}
    E=y+\alpha \dfrac{Q^{2}}{2gA^{2}}
\end{equation}

Derivando respecto a $y$ obtenemos:

\begin{equation}
    \dfrac{dE}{dy}=1-\dfrac{Q^{2}}{gA^{3}}\dfrac{dA}{dy}
\end{equation}

Luego para obtener el valor mínimo para $E$, se obtiene si $\dfrac{dE}{dy}=0$ para una sección prismática $\dfrac{dA}{dy}=B$, lo que se denomina ancho en la superficie.

Entonces para un canal trapezoidal:

\begin{equation}
    \begin{aligned}  
        A=(B_0+sy)y \qquad &\dfrac{dA}{dy}=B_0+2sy=B\\
        \text{Entonces} \quad 1-\dfrac{Q^{2}}{gA^{3}}B=0& \qquad 1-\dfrac{v^{2}}{gD}=0 \qquad \dfrac{D}{2}=\dfrac{v^{2}}{2g}\\
        \text{donde} \quad D\rightarrow \text{profundidad hidráulica}& \qquad D=\dfrac{A}{B}\\
        \text{Obteniendo} \quad \dfrac{d^{2}E}{dy^{2}}=0-\dfrac{d}{dy}\left(\dfrac{Q^{2}}{gA^{3}}\cdot B\right)=&\dfrac{-Q^{2}}{g}\left(\dfrac{-3B}{A^{4}}\dfrac{dA}{dy}+\dfrac{1}{A^{3}}\dfrac{dB}{dy}\right)\\
        \dfrac{d^{2}E}{dy^{2}}=\dfrac{Q^{2}}{g}\left(\dfrac{3B^{2}}{A^{4}}-\dfrac{1}{A^{3}}\dfrac{dB}{dy}\right)=&\dfrac{Q^{2}}{gA^{3}}\left(\dfrac{3B^{2}}{A}-\dfrac{dB}{dy}\right)
    \end{aligned}
\end{equation}

Esta siempre es positiva ya que $\dfrac{3b^{2}}{A}>\dfrac{dB}{dy}$ por lo tanto $E$ es mínima cuando $\dfrac{dE}{dy}=0$. Esta profundidad que se obtiene a partir de $\dfrac{D}{2}=\dfrac{v^{2}}{2g}\rightarrow$ es la profundiad crítica. \vspace{1ex}

Note que la profundidad hidráulica es dos veces al energía cinemática cuando el flujo es crítico (similar para el caso de un canal rectangular).

Analizando el número de Froude, para flujo crítico:

\begin{equation}
    \begin{aligned}
        Fr=1=\dfrac{v}{\sqrt{gD}} \rightarrow& \text{la cual se puede derivar de la ecuación para la energía mínima}\\ 
        &\dfrac{D}{2}=\dfrac{v^{2}}{2g} \rightarrow 1 = \dfrac{v}{\sqrt{gD}}=Fr
    \end{aligned} 
\end{equation}

para el caso de un canal de alta pendiente, con distribución de velocidad no uniforme en donde:

\begin{equation}
    E=dcos\theta+\alpha	\dfrac{Q^{2}}{2gA^{2}}
\end{equation}

Se puede obtener que

\begin{equation}
    Fr=1=\dfrac{v}{\sqrt{gd\dfrac{cos\theta}{\alpha}}} \quad \text{Donde} \quad \theta \quad \text{es el ángulo de inclinación del canal}
\end{equation}


\subsection{Fuerza específica}

Partiendo de la ecuación $Fr=\dfrac{Q^{2}}{gA}+\bar{z}A$ y derivando la ecuación respecto a $y$, se obtiene:

\begin{equation}
    \dfrac{dF_{s}}{dy}=-\dfrac{Q^{2}}{gA^{2}}\dfrac{dA}{dy}+\dfrac{d(\bar{z}A)}{dy}
\end{equation}

\textcolor{red}{\textbf{(SEGUNDA GRÁFICA)}} \vspace{1ex}

% \begin{figure}[H]
%     \centering
%     \includegraphics[width=0.6\linewidth]{}
%     \caption{Repre. momento}
%     \label{fig:familiacurvas}
% \end{figure} 

Donde $\bar{z}A$ representa el momento de $A$ respecto a la superficie de agua

El cambio en el  momento del área $(\Delta(\bar{z}A))$ debido a pequeños cambios en la profundidad del agua $\Delta y$ con respecto a la superficie del agua:

\begin{equation}
    \begin{aligned}
        \Delta(\bar{z}A)=A(\bar{z}+\Delta y)&-A\bar{z}+B\Delta y \cdot \dfrac{\Delta y}{2}\\ 
        \Delta(\bar{z}A)&=A \Delta y\\
        d(\bar{z}A)&=Ady\\
        \text{Reemplazando}& \qquad \dfrac{dF_{s}}{dy}=-\dfrac{Q^{2}}{gA^{2}}B+A\\
        \text{para} \quad \dfrac{dF_{s}}{dy}=0 \quad \dfrac{v^{2}}{g}=D \rightarrow \dfrac{v^{2}}{2g}=\dfrac{D}{2} &\rightarrow \text{Esta condición se satisface cuando el flujo es crítico}
    \end{aligned} 
\end{equation}

Como la última condición se satisface, por lo tanto $F_{s}$ es mínima cuando el flujo es crítico $(y=y_c)$ para probar que esto es un mínimo, $\dfrac{d^{2}F_{s}}{dy^{2}}=\dfrac{dA}{dy}+\dfrac{Q^{2}}{gA^{3}}\left(\dfrac{3B^{2}}{A}-\dfrac{dB}{dy}\right)$, luego como $\dfrac{3b^{2}}{A}>\dfrac{dB}{dy}$ el segundo término es siempre positivo al igual que $\dfrac{dA}{dy}=B$, por lo tanto $\dfrac{d^{2}F_{s}}{dy^{2}}\rightarrow$ es positiva, por lo que $F_s$ es un mínimo cuando $\dfrac{dF_{s}}{dy}=0$

\subsection{Aplicaciones del flujo crítico}

El flujo crítico establece una relación única entre el caudal y la profundidad del flujo. Teniendo en cuenta esto, \textcolor{blue}{muchas medidas de caudal} han sido desarrolladas. Por otro lado, recordemos que el flujo crítico garantiza \textcolor{blue}{el máximo caudal} a través de una sección. El flujo crítico se puede lograr:

\begin{itemize}
    \item Reduciendo la sección del flujo (ancho).
    \item Elevando el fondo del canal.
    \item Combinando las dos.
\end{itemize}

\subsubsection{Canal horizontal con ancho constante y elevación de fondo}

\textcolor{red}{\textbf{(TERCERA GRÁFICA)}} \vspace{1ex}

% \begin{figure}[H]
%     \centering
%     \includegraphics[width=0.6\linewidth]{}
%     \caption{Canal horizontal con ancho constante y elevación de fondo}
%     \label{fig:familiacurvas}
% \end{figure} 

Donde $\Delta z_{max}$ garantiza la ocurrencia de la profundidad crítica para una energía dada en 1, $\Delta z_{max}=E_{1}-E_{2}$ para un $\Delta z>\Delta z_{max}$ la profundidad en 1 debe aumentar y por lo tanto $E_{1}$ aumenta para garantizar que $\Delta z=E_{1}-E_{2}$

\subsubsection{Canal horizontal con ancho variable}

\textcolor{red}{\textbf{(CUARTA GRÁFICA)}} \vspace{1ex}

% \begin{figure}[H]
%     \centering
%     \includegraphics[width=0.6\linewidth]{}
%     \caption{Canal horizontal con ancho variable}
%     \label{fig:familiacurvas}
% \end{figure} 

En la medida que el ancho \textcolor{blue}{decrece}, la profundidad de agua \textcolor{blue}{decrece para}  flujo subcrítico aguas arriba. La profundidad de agua aumenta para una reducción del ancho para flujo supercrítico. \vspace{1ex}

Hay un limite del ancho en la reducción para una energía dada en 1 $(B_c)$, para un $B<B_{c}$ a veces aumenta \textcolor{blue}{el agua} en 1 y cambia la energía en $E_{1}$ para que $B=B_{c}$.


\begin{ejemplo}{1}

    Un puente será construido sobre un canal de $50m$ de ancho que transporta un caudal de $200m^{3}/s$ con una profundidad de $4.0m$. Con el fin de reducir el \textcolor{blue}{largo} del puente, cual es el ancho mínimo del canal que garantiza que la \textcolor{blue}{condición} aguas arriba del puente no cambiará. \vspace{2ex}

    \begin{solution} \vspace{1.5ex}

    Se tiene que: $Q=200m^{3}/s$, $b=50m$, $y_1=4.0m$

    \begin{equation}
        \begin{aligned}
            v=\dfrac{200 m^3/s}{50m \cdot 4m}=1m/s \qquad &E=4+\dfrac{1}{2\cdot9.81}=4,05m\\
            y_{c}=\dfrac{2}{3}E=\dfrac{2}{3}&(4.05)=2.7m\\
            q=\sqrt{gy^{3}_{c}}=\sqrt{9.81\cdot(2.7)^{3}}&=13.9 \dfrac{m^{3}/s}{m}\\
            q=\dfrac{Q}{B_c}=\dfrac{200}{13.9}=14.4m \rightarrow \quad &\text{Ancho mínimo que garantiza que la condición a.arriba no cambie}    
        \end{aligned} 
    \nonumber
    \end{equation}
    
    \end{solution}

\end{ejemplo}

\begin{ejemplo}{2}

    Para un canal rectangular \vspace{2ex}

    \begin{solution} \vspace{1.5ex}

    Se tiene que: $Q=250m^{3}/s$, $b=50m$, $y_1=5.0m$ \vspace{2ex}

    Para producir flujo crítico en este canal, determine:

    \begin{enumerate}
        \item El escalón en el fondo del canal para un ancho constante.
        \item La reducción en el ancho del canal.
        \item La combinación de escalon y ancho del canal.
    \end{enumerate}

    \begin{equation}
        \begin{aligned}
            v=\dfrac{Q}{A}=\dfrac{250 m^3/s}{50m \cdot 5m}=1m/s \qquad &E=y+\dfrac{v^{2}}{2g}=5+\dfrac{1}{2\cdot9.81}=5,05m\\
            y_{c}=\dfrac{2}{3}E=\dfrac{2}{3}&(5.05)=3.37m
        \end{aligned} 
    \nonumber
    \end{equation}

    a)  $$\Delta z_{max}=E_{1}-E_{2}=y_{1}-y_{c}=5-3.37=1.63m$$
    b)  $$q=\sqrt{gy^{3}_{c}}=\sqrt{9.81\cdot(3.37)^{3}}=19.38\dfrac{m^{3}/s}{m}$$
    $$q=\dfrac{Q}{B_{c}}=B_{c}=\dfrac{Q}{q}=\dfrac{250}{19.38}=12.9m$$
    c) \begin{center}
            Supongamos que $B_{c}=\dfrac{b}{2}=25m$, $B_{c}>12.9m$
        \end{center}
        $$q=\dfrac{Q}{B_{c}}=\dfrac{250}{25}=10\dfrac{m^{3}/s}{m} \qquad y_{c}=\sqrt[3]{\dfrac{q^{2}}{g}}$$
        $$y_{c}=\sqrt[3]{\dfrac{10^{2}}{9.81}}=2.17m$$
        $$E_{c}=2.17+\dfrac{10^{2}}{2\cdot9.81\cdot(2.17)^2}=3.25m$$
        $$\Delta z= 5.05-3.25=1.8m$$
    
    
    \end{solution}

\end{ejemplo}

\subsection{Localización del flujo crítico}

\subsubsection{Canal horizontal rectangular con aumento en el fondo}

$$\dfrac{dz}{dx}=(F^{2}_{r}-1)\dfrac{dy}{dx} \quad \text{para el flujo crítico} \quad F^{2}_{r}=1\rightarrow\dfrac{dz}{dx}=0$$
$$\dfrac{dz}{dx}=0 \quad \text{\textcolor{blue}{también} cuando} \quad\dfrac{dy}{dx}=0 \quad \text{por lo tanto} \quad \dfrac{dz}{dx}=0 \quad \text{en el punto más alto}$$
$$\dfrac{d^{2}z}{dx^{2}}=(F^{2}_{r}-1)\dfrac{d^{2}y}{dx^{2}}+2F_{r}\dfrac{dy}{dx}\dfrac{F_{r}}{dx}$$
$$\text{para el flujo crítico} \quad F_{r}=1\rightarrow\dfrac{d^{2}z}{dx^{2}}=2\dfrac{dy}{dx}\dfrac{dF_{r}}{dx}$$
$$y_{2}=y_{c}<y_{1}\rightarrow \dfrac{dy}{dx} (-)$$
$$F_{r_{2}}=F_{c}>F_{r}\rightarrow \dfrac{dF_{r}}{dx} (+)$$
$$\dfrac{d^{2}z}{dx^{2}}(-) \rightarrow \text{Se trata entonces de un máximo.} \quad y_c \quad\text{ocurre en el punto máximo} $$

\subsubsection{Canal horizontal con cambio en el ancho}
$$H=z+y+\dfrac{Q^{2}}{2gA^{2}}$$
$$\cancelto{0}{\dfrac{dH}{dx}}=\cancelto{0}{\dfrac{dz}{dx}}+\dfrac{dy}{dx}+\dfrac{d}{dx}\left(\dfrac{Q^{2}}{2gA^{2}}\right)=0 \qquad \dfrac{dy}{dx}+\dfrac{Q^{2}}{2g}\dfrac{d}{dx}\left(\dfrac{1}{b^{2}y^{2}}\right)=0$$
$$\dfrac{dy}{dx}+\dfrac{Q^{2}}{2g}\dfrac{(-2)}{b^{3}y^{2}}\dfrac{db}{dx}+\dfrac{Q^{2}}{2g}\dfrac{(-2)}{b^{2}y^{3}}\dfrac{dy}{dx}=0$$
$$\dfrac{dy}{dx}-\dfrac{Q^{2}}{y}\left(\dfrac{1}{b^{3}y^{2}}\dfrac{db}{dx}+\dfrac{1}{b^{2}y^{3}}\dfrac{dy}{dx}\right)=0$$
$$\dfrac{dy}{dx}=\dfrac{Q^{2}}{gy^{3}b^{2}}\left(\dfrac{y}{b}\dfrac{db}{dx}+\dfrac{dy}{dx}\right)=0 \qquad \dfrac{dy}{dx}-F^{2}_{r}\left(\dfrac{y}{b}\dfrac{db}{dx}+\dfrac{dy}{dx}\right)$$
$$\dfrac{dy}{dx}(1-F^{2}_{r})-F^{2}_{r}\dfrac{y}{b}\dfrac{db}{dx}=0 \qquad  \rightarrow para \qquad F_{r}=F_{r_{c}}=1\rightarrow\dfrac{db}{dx}=0$$

\subsection{Cálculo de la profundidad crítica}

\subsubsection{Canal con una única sección}

Partiendo de la ecuación general para un canal con pendiente $\theta$ y con velocidad no uniforme, se tiene:

\begin{equation}
    F_{r}=\dfrac{v}{\sqrt{gD\dfrac{cos\theta}{\alpha}}}=1
\end{equation}

Como  $Q=vA\rightarrow v=\dfrac{Q}{A}$, reemplazando, se tiene que:

\begin{equation}
    \dfrac{Q}{A\sqrt{gD\dfrac{cos\theta}{\alpha}}}=1
\end{equation}

Donde $D=\dfrac{A}{B}\rightarrow \text{profundidad hidráulica}$
$$B=\dfrac{dA}{dy}\rightarrow \text{Ancho del canal en la superficie}$$
$$\text{Haciendo} \quad A\sqrt{D}=\dfrac{Q}{\sqrt{gD\dfrac{cos\theta}{\alpha}}}$$

La profundidad crítica $y_c$ se encuentra resolviendo la ecuación anterior. Note que $A\sqrt{D}=f(y_c \quad \text{\textcolor{blue}{y ??}})$ y $\dfrac{Q}{\sqrt{gD\dfrac{cos\theta}{\alpha}}}=cte$ para \textcolor{blue}{and?} de sección única, para un caudal dado, se tiene un solo valor de $y_{c}$ \textcolor{blue}{de pa de ven?} esta ecuación. Esta ecuación \textcolor{blue}{se puede reducir} mediante:

\subsubsection{Curva de diseño}

$$z_{c}=A\sqrt{D}=\text{factor de sección para $y_c$}$$ 
$$\textcolor{blue}{v_{s}}$$
$$\textcolor{blue}{y_{c}}$$

Comun para \textcolor{blue}{canales} trapezoidales y circulares

\section{Métodos numéricos}

Los principales métodos utilizados para encontrar las raices de una función implícita $f(y)=0$ son:

\begin{enumerate}
    \item Método de la sección.
    \item Método de la bisección.
    \item Método de Newton-Raphson. \textit{(mejor)}
\end{enumerate}

$$A\sqrt{D}-\dfrac{Q}{\sqrt{g\dfrac{cos\theta}{\alpha}}}=0=f(y)$$
$$A^{\frac{3}{2}}B^{\frac{-1}{2}}-\dfrac{Q}{\sqrt{g\dfrac{cos\theta}{\alpha}}}=0$$

El método de Newton-Raphson requiere el cálculo de $\dfrac{df}{dy}$, se tiene entonces que:


$$\dfrac{df}{dy_{c}}=\dfrac{3}{2}A^{\frac{1}{2}}B^{\frac{-1}{2}}\dfrac{dA}{dy}-\dfrac{1}{2}B^{\frac{-3}{2}}\dfrac{dB}{dy}A^{\frac{3}{2}}\dfrac{dB}{dy}=f(f_{c})$$


En ese orden de ideas para el cálculo de $\dfrac{dB}{dy}$:


\begin{itemize}
    \item Canal \textcolor{blue}{????}:

    $$\dfrac{dB}{dy_{c}}\approx\dfrac{B(y_{c}+h)-B(y_c-h)}{2h}$$

    Donde: $h\rightarrow$ \textcolor{blue}{???} pequeño en $y=10^{-6}$

    \textcolor{red}{\textbf{(QUINTA GRÁFICA)}} \vspace{1ex}

    \item Para canal prismático y \textcolor{blue}{????} $B=f(y_{c})$

    $$\dfrac{dB}{dy_{c}}=2s \qquad \text{(Para canal trapezoidal $B=B_{0}+2sy_{c}$)}$$

\end{itemize}

El método de Newton Rhapson consiste \textcolor{blue}{????} en:

$$y^{t+1}_{c}=y^{t}_{c}-\dfrac{f(y_c)}{f'(y_c)}$$

El proceso iterativo para cuando $|y^{t+1}-y^{t+}|\leq tolerancia$.

El valor inicial para se puede obtener suponiendo que el canal es rectangular, por lo que:

$$A\sqrt{D}=\dfrac{Q}{\sqrt{g\dfrac{cos\theta}{\alpha}}}\rightarrow by^{\frac{3}{2}}=\dfrac{Q}{\sqrt{g\dfrac{cos\theta}{\alpha}}}$$
$$y_{c}=\left(\dfrac{Q}{b\sqrt{g\dfrac{cos\theta}{\alpha}}}\right)^{\frac{2}{3}}$$

\begin{ejemplo}{1}

Canal trapezoidal

$Q=30m^{3}/s$, $B_{0}=10m$, $s=2, \theta \approx0, \alpha=1$

Calculo la profundidad crítica:

\begin{solution}
    Sabiendo que para un canal trapezoidal se tiene que, $B=B_{0}+2sy$

    $$z_{c}=A\sqrt{D}=\dfrac{Q}{\sqrt{g}}$$
    $$\dfrac{Q}{\sqrt{g}}=\dfrac{30}{\sqrt{9.81}}=9.58$$
    $$A=\dfrac{1}{2}(B_{0}+2sy_{c}+B_{0})y_{c}=(B_{0}+sy_{c})y_{c}=(10+2y_{c})y_{c}$$
    $$B=\dfrac{dA}{dy}=B_{0}+2sy_{c}=10+4y_{c}$$
    $$D=\dfrac{A}{B}=\dfrac{(10+2y_{c})y_{c}}{10+4y_{c}}$$
    $$(10+2y_{c})y_{c}\sqrt{\dfrac{(10+2y_{c})y_{c}}{10+4y_{c}}}-9.58=0=f(y_c)$$

    Utilizando un solver para la ecuación anterior $(f(y_c))$ o utilizando el método de Newton Raphson, se tiene que $y_c=0.91m$, si se usa el método de Newton Raphson. $y_c$ para $t=0$

    $$y^{t=0}_{c}=\left(\dfrac{30}{10\sqrt{9.81}}\right)^\frac{2}{3}=0.17m$$
    
\end{solution}
    
\end{ejemplo}

\subsection{Método de la Bisección}

    \begin{enumerate}
        \item Elegir un intervalo de $y_{c}$, $a<y_{c}<b$ tal que $f(a)f(b)<0$ (Signo opuesto).
        \item Calcular el punto medio del intervalo $c=\dfrac{a+b}{2}$
        \item Evaluar $f$ en $c$, si $f(c)$

        Si $f(c)\approx0$ $y_c=c$
        Si $f(a)f(c)<0\rightarrow a<y_c<b \rightarrow b=c$
        Si $f(c)f(b)<0\rightarrow c<y_c<b \rightarrow a=c$

        \textcolor{red}{\textbf{(SEXTA GRÁFICA)}} \vspace{1ex}

        \item El \textcolor{blue}{???} de \textcolor{blue}{????} cuando $f(c)\leq 10^{-6}$ (tolerancia).
    \end{enumerate}


\begin{ejemplo}{2}
    Canal circular

    $Ø=8ft, m=1ft/milla, Q=100 cfs \quad (\text{cubic feet per second}), \alpha=1$

    \begin{solution}
    
    Primeramente, se tiene la conversión:

    $$m=1\dfrac{ft}{milla}\cdot\dfrac{1milla}{5280ft}=0.0002$$

    Calculando el valor de $y_c$, se tiene:

    $$\theta=arctan(0.0002)=0.011  \qquad \theta \approx 1$$
    $$A\sqrt{D}=\dfrac{Q}{\sqrt{g}}\rightarrow A\sqrt{D}-\dfrac{Q}{\sqrt{g}}=0=f(y_c)$$
    $$\dfrac{Q}{\sqrt{g}}=\dfrac{100}{\sqrt{32.2}}=17.62$$
    $$A=\dfrac{1}{8}(\theta-sin\theta)Ø^{2}$$
    $$B=Øsin\dfrac{1}{2}\theta$$
    $$A^{3/2}B^{-1/2}-17.62=0$$
    $$\left(\dfrac{1}{8}(\theta-sin\theta)Ø^{2})\right)^{3/2}\left(Øsin\dfrac{1}{2}\theta\right)^{-1/2}-17.62=0$$
    $$y_c=\dfrac{Ø}{2}(1-cos\theta/2)=$$
    \end{solution}

    
\end{ejemplo}

    





    

















\end{document}
 